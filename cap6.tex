\chapter{Conclusão}
\label{cha:conclusao}
% \todo[inline]{2 ou 3 parágrafos para concluir o que foi apresentado neste trabalho.}
Este trabalho apresentou a relação entre o aumento de núcleos e o consumo de
energia nos processadores atuais. A energia necessária para alcançar supercomputadores
\textit{Exascale} é muito alta, se tornando necessária a utilização de
novas técnicas e processadores energeticamente eficiêntes. Contudo, esses
processadores apresentam dificuldades, trazendo problemas para o desenvolvimento
de aplicações e motivando a utilização de \fws sobre esses ambientes.

Contudo, a adaptação do \fw para o processador \textit{manycore} \mppa é uma
tarefa desafiadora. Problemas de sincronização e comunicação deverão ser
tratados de forma especial, considerando a relação do tempo que o \fw gasta
comunicando e realizando a computação sobre os dados de entrada. Desta forma,
encontrar o melhor \textit{trade-off} entre o número de comunicações e a
quantidade de computação realizada \textit{clusters} é essencial.
