\chapter{Trabalhos Relacionados}

A proposta deste trabalho está diretamente relacionada a diversos outros trabalhos de pesquisa.
A seguir, serão citados alguns trabalhos de pesquisa que fazem uso de esqueletos paralelos em
arquiteturas \textit{manycore}. Além disso, serão destacados alguns trabalhos de pesquisa sobre
o \mppa.

% \todo[inline]{Aqui, acho que poderia subdividir os trabalhos relacionados em subseções, de acordo com os assuntos que eles tratam.}

\subsection{Esqueletos paralelos e padrão \textit{stencil}}
\emph{Buono}~\etal~\cite{buono13} portaram um \fw baseado em esqueletos paralelos,
chamado de \emph{FastFlow}, para o processador \textit{manycore} \emph{TilePro64}.
Esse processador possui 64 núcleos de processamento idênticos, interconectados
por uma malha da \noc. O \fw \emph{FastFlow} provê padrões de \textit{design}
customizáveis, como, por exemplo, \textit{pipelines} e \textit{task farms},
que podem ser compostas para formar outros esqueletos, como \textit{map} e
\textit{reduce}.

De forma similar, \emph{Thorarensen}~\etal~\cite{thoraransen16} apresentaram um
novo \textit{back-end} do \fw \emph{SkePU} para o processador \textit{manycore}
de baixa potência \emph{Myriad2}. Esse processador possui uma arquitetura
heterogênea, tendo como alvo dispositivos com limites em questão de energia.
O \fw \emph{SkePU} provê uma interface de programação para esqueletos paralelos
como o \textit{map}, \textit{reduce}, e \textit{stencil}, com suporte para
diferentes \textit{back-ends}, incluindo processadores multicores e \gpus.

%------Trabalho relacionado sobre técnicas de tiling sobre o padrão estêncil.
%------Deixar para mencionar sobre trabalhos relacionados à isso.
\emph{Lutz}~\etal~\cite{lutz13} utilizaram técnicas de \textit{tiling} em
computações \textit{stencil} para lidar com a capacidade limitada de memória
de \gpus em ambientes multi-\gpu, utilizando as memórias das \gpus
coletivamente. De forma similar, \emph{Gysi}~\etal~\cite{gysi15} propuseram um \fw
para otimizações automáticas de \textit{tiling} em computações \textit{stencil}
situadas em um ambiente híbrido \cpu{}-\gpu.

%------------------------------------------

\subsection{Processadores \textit{manycore} de baixo consumo de energia}
Alguns trabalhos surgiram recentemente com intuito de avaliar o uso de
processadores \emph{manycore} em CAD, além de discutir os desafios do
desenvolvimento de aplicações para esses processadores.
Em~\cite{SCCEnergy:2012}, os autores compararam o desempenho e o consumo
energético de um processador \emph{manycore} experimental da Intel denominado
\emph{Single-Chip Cloud Computer} (SCC) com outros tipos de processadores e
GPUs. Para realizar essa análise, os autores utilizaram um conjunto de
aplicações paralelas implementadas em Charm++~\cite{Charm:2012}. Os resultados
obtidos com o Intel SCC mostraram que \emph{manycores} são uma alternativa
viável, apresentando bom desempenho e baixo consumo energético.
Em~\cite{MPPA-1:2013}, os autores avaliaram o desempenho do processador
\emph{manycore} MPPA-256 no contexto de aplicações de decodificação de vídeo. Os
resultados mostraram que o desempenho do MPPA-256 é comparável ao desempenho de
processadores Intel atuais em uma decodificação de vídeo no padrão H.264,
consumindo 6 vezes menos energia.

Trabalhos recentes revelaram o desempenho e consumo energético do processador
MPPA-256, comparando-o a outros processadores \textit{multicore} de propósito
geral e embarcados, no contexto de diferentes aplicações
científicas~\cite{Castro-SBAC-PAD:2014,Castro-IA3:2013,Castro-IA3-JPDC:2014}. Os
resultados mostraram que o processador \emph{manycore} MPPA-256 apresenta em
alguns casos desempenho superior a processadores \emph{multicore} Intel Xeon 2.4
GHz com 8 \emph{cores}, além de um consumo de energia de até 13 vezes menor em
relação ao mesmo processador. Um outro trabalho recentemente publicado realizou
uma análise comparativa de desempenho e consumo de energia entre processadores
\emph{multicore} Intel de alto desempenho e ARM~\cite{Castro-Padoin-IET:2015}.
Os resultados mostraram que, apesar da potência dos processadores ARM ser pelo
menos 10 vezes menor que a dos processadores Intel de alto desempenho, o consumo
de energia nem sempre será melhor, sendo dependente das características da carga
de trabalho a ser executada.

\emph{Morari}~\etal~\cite{Valero:2012} propuseram uma implementação otimizada do
\textit{radix sort} para o processador \textit{manycore} Tilera TILEPro64. Os
resultados mostraram que a solução para o TILEPro64 provê uma melhor eficiência
energética em relação a um processador \textit{multicore} de propósito geral, como
o Intel Xeon W5590, e em relação a uma \gpu NVIDIA Tesla C2070.

Mais especificamente, \emph{Francesquini}~\etal~\cite{Castro-IA3-JPDC:2014} analisaram
três diferentes classes de aplicações (CPU-\textit{bound}, \textit{memory-bound}
e híbrida) usando plataformas paralelas, como o \mppa, e um multiprocessador \numa de
192 núcleos e 24 nós. Mostrou-se que arquiteturas \textit{manycore} podem ser muito
competitivas, mesmo se a aplicação é, naturalmente, irregular. Os resultados mostraram
que o \mppa pode obter um desempenho maior (e um consumo de energia menor) que um
processador \textit{multicore} de propósito geral (Intel Xeon E5-4640) em um ambiente
com cargas de trabalho variadas e \cpu{}-\textit{bound}. Todavia, em um ambiente com cargas
de trabalho \textit{memory-bound}, o processador \numa obteve um melhor desempenho
em relação ao \mppa, apesar de apresentar também um maior consumo de energia.
Entre as plataformas avaliadas, o \mppa apresentou a melhor eficiência
energética, reduzindo a energia consumida em aplicações \cpu{}-\textit{bound},
híbridas e \textit{memory-bound} em 6.9x, 6.5x e 3.8x, respectivamente.

% Trabalhos recentes que propuseram APIs e ambientes de execução para
% \emph{manycores}. Em~\cite{TSHMEM:2013}, os autores propuseram uma adaptação
% Espaço de Endereçamento Global Particionado (\emph{Partitioned Global Address
%     Space} -- PGAS) para simplificar o desenvolvimento de aplicações paralelas
% para os processadores \emph{manycore} TILE-Gx e TILEPro. Mais precisamente, os
% autores utilizaram a biblioteca OpenSHMEM como base para a proposta,
% utilizando-a como uma camada de abstração para as bibliotecas fornecidas pelo
% fabricante dos processadores. Com isso, aplicações atualmente implementadas
% utilizando a API OpenSHMEM podem ser executadas nos processadores
% \emph{manycore} da linha TILE sem que haja a necessidade de modificações no
% código.
