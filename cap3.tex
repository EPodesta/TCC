\chapter{Trabalhos Relacionados}

Alguns trabalhos surgiram recentemente com intuito de avaliar o uso de processadores \emph{manycore} em CAD, além de discutir os desafios do desenvolvimento de aplicações para esses processadores.
Em~\cite{SCCEnergy:2012}, os autores compararam o desempenho e o consumo energético de um processador \emph{manycore} experimental da Intel denominado \emph{Single-Chip Cloud Computer} (SCC) com outros tipos de processadores e GPUs. Para realizar essa análise, os autores utilizaram um conjunto de aplicações paralelas implementadas em Charm++~\cite{Charm:2012}. Os resultados obtidos com o Intel SCC mostraram que \emph{manycores} são uma alternativa viável, apresentando bom desempenho e baixo consumo energético.
Em~\cite{MPPA-1:2013}, os autores avaliaram o desempenho do processador \emph{manycore} MPPA-256 no contexto de aplicações de decodificação de vídeo. Os resultados mostraram que o desempenho do MPPA-256 é comparável ao desempenho de processadores Intel atuais em uma decodificação de vídeo no padrão H.264, consumindo 6 vezes menos energia.

Trabalhos recentes revelaram o desempenho e consumo energético do processador MPPA-256, comparando-o a outros processadores \textit{multicore} de propósito geral e embarcados, no contexto de diferentes aplicações científicas~\cite{Castro-SBAC-PAD:2014,Castro-IA3:2013,Castro-IA3-JPDC:2014}. Os resultados mostraram que o processador \emph{manycore} MPPA-256 apresenta em alguns casos desempenho superior a processadores \emph{multicore} Intel Xeon 2.4 GHz com 8 \emph{cores}, além de um consumo de energia de até 13 vezes menor em relação ao mesmo processador. Um outro trabalho recentemente publicado realizou uma análise comparativa de desempenho e consumo de energia entre processadores \emph{multicore} Intel de alto desempenho e ARM~\cite{Castro-Padoin-IET:2015}. Os resultados mostraram que, apesar da potência dos processadores ARM ser pelo menos 10 vezes menor que a dos processadores Intel de alto desempenho, o consumo de energia nem sempre será melhor, sendo dependente das características da carga de trabalho a ser executada.

Trabalhos recentes que propuseram APIs e ambientes de execução para \emph{manycores}. Em~\cite{TSHMEM:2013}, os autores propuseram uma adaptação Espaço de Endereçamento Global Particionado (\emph{Partitioned Global Address Space} -- PGAS) para simplificar o desenvolvimento de aplicações paralelas para os processadores \emph{manycore} TILE-Gx e TILEPro. Mais precisamente, os autores utilizaram a biblioteca OpenSHMEM como base para a proposta, utilizando-a como uma camada de abstração para as bibliotecas fornecidas pelo fabricante dos processadores. Com isso, aplicações atualmente implementadas utilizando a API OpenSHMEM podem ser executadas nos processadores \emph{manycore} da linha TILE sem que haja a necessidade de modificações no código.
