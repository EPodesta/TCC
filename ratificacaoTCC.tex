\documentclass[12pt]{article}
\usepackage[a4paper,margin=2cm]{geometry}

\usepackage[T1]{fontenc}
\usepackage[utf8]{inputenc}

\usepackage{mathptmx}
\usepackage{tabularx}
\usepackage{multirow}
\usepackage{makecell}
\usepackage{pgfgantt}
\usepackage{graphics}
\usepackage{graphicx}
\begin{document}

\pagestyle{empty}

\begin{centering}

    \textbf{DEPARTAMENTO DE INFORMÁTICA E ESTATÍSTICA -- CTC -- UFSC}

    \textbf{RATIFICAÇÃO DE PLANO DE TRABALHO DO SEMESTRE \\ PARA DESENVOLVIMENTO DE TCC}

\end{centering}


\vspace{1em}
\setlength\extrarowheight{5pt}
\begin{tabular}{l l l}
    \textbf{Disciplina:} & ( ) TCC 1  &  ( \textbf{X} ) TCC 2\\
    \textbf{Curso:}      & ( \textbf{X} ) CCO & ( ) SIN \\
    \textbf{Autor:}      & Emmanuel Podestá Junior &\\
\end{tabular}
\vspace{0.5cm}
\\
\begin{tabular}{l l}
\vspace{0.5cm}
     \textbf{Título:} & \makecell{PSkel-MPPA: Uma Adaptação do Framework PSkel para o \\Processador Manycore MPPA-256}\\
    \textbf{Professor responsável:} & Prof. Dr. Márcio Bastos Castro\\
\end{tabular}


\vspace{1em}
{\large \textbf{Objetivos}}
\\

\textbf{Objetivo geral:}
O objetivo do TCC é abordar as características do PSkel e adaptá-lo para o processador \textit{manycore} MPPA-256. Essa adaptação permitirá o uso da abstração provinda do \textit{framework} para o desenvolvimento de aplicações \textit{stencil} para o processador, reduzindo assim as dificuldades encontradas no desenvolvimento de aplicações para o processador.

\vspace{1em}
{\large \textbf{Cronograma}}

%\begin{tabularx}{\linewidth}{|X|*{8}{c|}}
%    \hline
%    \multicolumn{1}{|c|}{\multirow{2}{*}{Etapas}} & \multicolumn{8}{|c|}{Meses}\\ \cline{2-9}
%    & mar & abr & mai & jun & jul & ago & set & out \\ \hline

%    Estudar classes de complexidade computacional
%    &  x  &     &     &     &     &     &     &     \\ \hline

%    Estudar classes de complexidade de circuitos
%    &  x  &  x  &  x  &     &     &     &     &     \\ \hline

%    Estabelecer relações de equivalência entre as classes de complexidade
%    &  x  &  x  &  x  &  x  &     &     &     &     \\ \hline

%    Rever demonstrações de limites inferiores para \mbox{certas} funções recursivas
%    &     &  x  &  x  &  x  &  x  &  x  &     &     \\ \hline

%    Explorar a relação com o problema $P$ versus $NP$
%    &     &  x  &     &     &     &  x  &     &     \\ \hline

%    Documentar o aprendido
%    &  x  &  x  &  x  &  x  &  x  &  x  &  x  &  x  \\ \hline

%\end{tabularx}

\begin{figure}[!h]
	\begin{center}

		\begin{ganttchart}[
                x unit=1.6cm,
				y unit title=1cm,
				y unit chart=1cm,
				hgrid,
				vgrid={{dotted, dotted, dotted, dotted, dotted, dotted}},
				% title label font=\3scriptsize,
				title/.append style={fill=gray!30},
				title height=1,
				bar/.append style={fill=gray!30,rounded corners=2pt},
				bar label font=\scriptsize,
				group label font=\scriptsize,
			]{1}{6}
			\gantttitle{\textbf{Meses}}{6} \\
			\gantttitle{\textbf{2018}}{6} \\
			\gantttitlelist{1,2,3,4,5,6}{1} \\
            \ganttbar{1. Escrita aprofundada da proposta e implementação}{1}{3} \\
            \ganttbar{2. Escrita dos experimentos realizados}{3}{4} \\
            \ganttbar{3. Apresentação do TCC}{5}{5} \\
            \ganttbar{4. Revisão e entrega final do TCC}{6}{6} \\

		\end{ganttchart}
	\end{center}
    \label{tab:cronograma}
\end{figure}


\begin{centering}

    \fbox{\begin{minipage}[c][6em][c]{0.7\textwidth}
        {\center \textbf{Preenchimento pelo Professor responsável pelo TCC}\\[1em]}

        \qquad $(\quad)$ \ Ciente e de acordo.

        \qquad Data: \_\_ / \_\_ / \_\_
    \end{minipage}}

\end{centering}
\end{document}
