\chapter{Cronograma}
\label{cha:cronograma}
% \todo[inline]{Descrever resumidamente as atividades.}

Esta seção apresenta o cronograma de atividades proposto para a conclusão do projeto apresentado neste documento.
A Figura~\ref{fig:cronograma} apresenta o cronograma previsto, com atividades nomeadas de A1 até A7. A descrição de cada
uma dessas atividades é mostrada a seguir.

  \begin{figure}[ht]
    \begin{center}
	\resizebox{\textwidth}{!}{
             \begin{ganttchart}[
               x unit = 1cm,
               y unit title=0.4cm,
               y unit chart=0.6cm,
               hgrid,
               vgrid={{dotted, dotted, dotted, black}},
               title label font=\scriptsize,
               title/.append style={fill=gray!30},
               title height=1,
               bar/.append style={fill=gray!30,rounded corners=2pt},
               bar label font=\scriptsize,
               group label font=\scriptsize,
               ]{1}{16}
             	% \gantttitle{\textbf{2017}}{4}
                \gantttitle{\textbf{2018}}{16}\\
             	% \gantttitle{\textbf{Nov}}{2}
             	\gantttitle{\textbf{Dez}}{2}
        	 \gantttitle{\textbf{Jan}}{2}
        	 \gantttitle{\textbf{Fev}}{2}
        	 \gantttitle{\textbf{Mar}}{2}
        	 \gantttitle{\textbf{Abr}}{2}
        	 \gantttitle{\textbf{Mai}}{2}
        	 \gantttitle{\textbf{Jun}}{2}
        	 \gantttitle{\textbf{Jul}}{2} \\
        	 % \gantttitle{\textbf{Ago}}{2}
        	 % \gantttitle{\textbf{Set}}{2}
        	 % \gantttitle{\textbf{Out}}{2}
        	 % \gantttitle{\textbf{Nov}}{2}\\

             \ganttbar{A1}{1}{4} \\
             \ganttbar{A2}{3}{8} \\
             \ganttbar{A3}{7}{10} \\
             \ganttbar{A4}{8}{12} \\
             \ganttbar{A5}{13}{13} \\
             \ganttbar{A6}{14}{14} \\
             \ganttbar{A7}{15}{15}
             % \ganttbar{A8}{24}{24} \\
             % \ganttbar{A9}{25}{25} \\
             % \ganttbar{A10}{26}{26}
             \end{ganttchart}
     }
%  \end{adjustwidth}
     \caption{Cronograma de atividades.}\label{fig:cronograma}
  \end{center}
\end{figure}
% ---

\begin{itemize}
    \item \textbf{A1: Revisão aprofundada do estado da arte e prática.} Nesta etapa será
        realizada uma revisão mais aprofundada do estado da arte, buscando mais
        conteúdo sobre o tema da proposta, e uma base conceitual sobre abordagens de
        comunicação entre a \cpu e aceleradores.
    % \item \textbf{A2: Elaboração da proposta.} Nesta etapa será realizada a
        % elaboração da proposta
    \item \textbf{A2: Implementação da proposta.} Nesta etapa será realizada a
        implementação da proposta descrita no Capítulo~\ref{cha:proposta},
        buscando, também, estudar outras implementações para alcançar um bom
        desempenho e consumo de energia no processador \mppa.
    \item \textbf{A3: Realização de experimentos.} Nesta etapa serão realizados
        experimentos comparativos entre a implementação proposta sobre o \mppa e
        outros processadores. Os experimentos irão avaliar o desempenho e o consumo de
        energia da implementação proposta em relação à outros processadores.
    \item \textbf{A4: Escrita do rascunho do TCC II.} Nesta etapa será realizada
        a escrita do rascunho do TCC II, apresentando os experimentos desenvolvidos e a
        implementação completa da proposta.
    \item \textbf{A5: Preparação da defesa pública.} Nesta etapa será realizada
        a preparação de \textit{slides} e ensaio da defesa pública.
    \item \textbf{A6: Defesa pública.} Nesta etapa será realizada a defesa do
        projeto desenvolvido, mostrando os resultados obtidos e suas conclusões.
    \item \textbf{A7: Correções e entrega da versão final do TCC.} Nesta etapa
        será realizada as correções solicitadas pelos membros da banca, buscando
        melhorar a versão final do TCC.
\end{itemize}
% \begin{adjustwidth}{2.5cm}{}

