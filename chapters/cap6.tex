\chapter{Conclusão}
\label{cha:conclusao}
% \todo[inline]{2 ou 3 parágrafos para concluir o que foi apresentado neste trabalho.}
Este trabalho discutiu inicialmente a relação entre o aumento de núcleos e o consumo de
energia nos processadores atuais. Como mostrado, a energia necessária para alcançar supercomputadores
\textit{Exascale} é muito alta, se tornando necessária a utilização de
novas técnicas e processadores energeticamente eficientes. Contudo, esses
processadores apresentam dificuldades de programação, tais como a existência de
um modelo de programação híbrido, capacidade limitada de memória no
\textit{chip}, ausência de coerência de \textit{cache}, entre outros. Essas
dificuldades trazem problemas para o desenvolvimento
de aplicações e motivam a utilização de \fws sobre esses ambientes.

% A adaptação do \fw para o processador \textit{manycore} \mppa é uma
% tarefa desafiadora. Problemas de sincronização e comunicação deverão ser
% tratados de forma especial, considerando a relação do tempo que o \fw gasta
% comunicando e realizando a computação sobre os dados de entrada. Desta forma,
% encontrar o melhor balanceamento entre o número de comunicações e a
% quantidade de computação realizada \textit{clusters} é essencial.

Neste trabalho foi proposta uma adaptação de um \textit{framework} para
desenvolvimento de aplicações \stencil iterativas, denominado \pskel, para
processador \mppa. A solução proposta permite esconder detalhes de baixo nível
do \mppa, simplificando significativamente o desenvolvimento de aplicações
\stencil nesse processador. Os resultados mostraram que a solução proposta
apresenta boa escalabilidade. Além disso, foi observado uma redução
significativa no tempo de execução e no consumo de energia das aplicações no
\mppa ao se utilizar a técnica de \textit{tiling} trapezoidal. Isso se deve,
principalmente, à redução do sobrecusto de comunicações e sincronizações de
\textit{tiles}.

A aplicação \textit{Fur} apresentou os melhores resultados de escalabilidade
dentre as 3 aplicações estudas, obtendo um \textit{speedup} de $14$x em relação
à apenas um \textit{cluster}. Analisando experimentos executados sobre a
adaptação pôde-se perceber uma relação entre a quantidade de computação
realizada pelo \textit{kernel} da aplicação e o \textit{speedup} obtido. Por
fim, experimentos comparativos entre o \mppa e o processador Intel Broadwell
mostraram que a solução proposta para o \mppa apresenta uma eficiência
energética superior apesar de um tempo de execução superior.

Como trabalhos futuros, pretende-se estudar formas de reduzir ainda mais os
sobrecustos de comunicação através do uso de técnicas de \textit{software
    prefetching}. Esta técnica possibilitará a construção de \textit{tiles}
alargados durante as computações de outros \textit{tiles} pelos
\textit{clusters}. Portanto, ao terminar a computação, outro \textit{tile}
alargado estará esperando para ser computado. Desta forma, será possível
diminuir o impacto da construção de \textit{tiles} alargados e seu envio sobre o
desempenho.
Além disso, pretende-se realizar experimentos com outros
\textit{benchmarks} e aplicações que utilizam estruturas tridimensionais. Por
fim, pretende-se realizar comparações de desempenho e consumo de energia com
outros processadores embarcados.

%\todo[inline]{o software prefetching ficou meio solto ali. o que ele faz e por que melhoraria o resultado?}
