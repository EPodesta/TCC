\documentclass[12pt]{article}

\usepackage[brazilian]{babel}
\usepackage{sbc-template}
\usepackage{times}

\usepackage[utf8]{inputenc}
\usepackage{xspace}

\usepackage[acronym,nowarn]{glossaries}
\newacronym{MPI}{MPI}{\textit{Message Passing Interface}}
\newcommand{\mpi}{\gls{MPI}\xspace}

\newacronym{openMP}{OpenMP}{\textit{Open Multi-Processing}}
    \newcommand{\openMP}{\gls{openMP}\xspace}

\newacronym{api}{API}{\textit{Application Programming Interface}}
    \newcommand{\api}{\gls{api}\xspace}
    \newcommand{\apis}{\glspl{api}\xspace}

\newacronym{cpu}{CPU}{\textit{Central Processing Unit}}
    \newcommand{\cpu}{\gls{cpu}\xspace}

    \newacronym{cpus}{CPUs}{\textit{Central Processing Units}}
    \newcommand{\cpus}{\gls{cpus}\xspace}

\newacronym{flops}{Flops}{\textit{Floating-point operations per second}}
    \newcommand{\flops}{\gls{flops}\xspace}

\newacronym{cnoc}{C-NoC}{\textit{Control NoC}}
    \newcommand{\cnoc}{\gls{cnoc}\xspace}

\newacronym{dnoc}{D-NoC}{\textit{Data NoC}}
    \newcommand{\dnoc}{\gls{dnoc}\xspace}

\newacronym{hpc}{HPC}{\textit{High Performance Computing}}
    \newcommand{\hpc}{\gls{hpc}\xspace}

\newacronym{mpsoc}{MPSoC}{\textit{Multiprocessor System-on-Chip}}
    \newcommand{\mpsoc}{\gls{mpsoc}\xspace}

\newacronym{noc}{NoC}{\textit{Network-on-Chip}}
    \newcommand{\noc}{\gls{noc}\xspace}
    \newcommand{\nocs}{\glspl{noc}\xspace}

\newacronym{pe}{PE}{\textit{Processing Element}}
    \newcommand{\pe}{\gls{pe}\xspace}
    \newcommand{\pes}{\glspl{pe}\xspace}

\newacronym{rm}{RM}{\textit{Resource Manager}}
    \newcommand{\rman}{\gls{rm}\xspace}
    \newcommand{\rmans}{\glspl{rm}\xspace}

\newacronym{smp}{SMP}{\textit{Symmetric Multiprocessing}}
    \newcommand{\smp}{\gls{smp}\xspace}

\newacronym{spmd}{SPMD}{\textit{Single Program, Multiple Data}}
    \newcommand{\spmd}{\gls{spmd}\xspace}

    \newacronym{simd}{SIMD}{\textit{Single Instruction, Multiple Data}}
    \newcommand{\simd}{\gls{simd}\xspace}

\newacronym{vliw}{VLIW}{\textit{Very Long Instruction Word}}
    \newcommand{\vliw}{\gls{vliw}\xspace}

\newacronym{gpu}{GPU}{\textit{Graphics Processing Unit}}
    \newcommand{\gpu}{\gls{gpu}\xspace}
    \newcommand{\gpus}{\glspl{gpu}\xspace}

\newacronym{rapl}{RAPL}{\textit{Running Average Power Limit}}
    \newcommand{\rapl}{\gls{rapl}\xspace}

\newacronym{lpddr}{LPDDR3}{\textit{Low Power Double Data Rate 3}}
    \newcommand{\lpddr}{\gls{lpddr}\xspace}

\newacronym{io}{E/S}{Entrada e Saída}
    \newcommand{\io}{\gls{io}\xspace}

\newacronym{ipc}{IPC}{\textit{Inter-Process Communication}}
   \newcommand{\ipc}{\gls{ipc}\xspace}

\newacronym{numa}{NUMA}{\textit{Non-Uniform Memory Access}}
	\newcommand{\numa}{\gls{numa}\xspace}

\newacronym{ccnuma}{CC-NUMA}{\textit{Cache-Coherent Non-Uniform Memory Access}}
\newcommand{\ccnuma}{\gls{ccnuma}\xspace}

\newacronym{ncnuma}{NC-NUMA}{\textit{No Cache Non-Uniform Memory Access}}
\newcommand{\ncnuma}{\gls{ncnuma}\xspace}

\newacronym{soc}{SoC}{\textit{System-on-Chip}}
\newcommand{\soc}{\gls{soc}\xspace}

\newacronym{cmp}{CMP}{\textit{Chip Multiprocessor}}
\newcommand{\cmp}{\gls{cmp}\xspace}

\newacronym{cmps}{CMPs}{\textit{Chip Multiprocessors}}
\newcommand{\cmps}{\gls{cmps}\xspace}


\newacronym{uma}{UMA}{\textit{Uniform Memory Access}}
    \newcommand{\uma}{\gls{uma}\xspace}

\newacronym{ram}{RAM}{\textit{Random-Access Memory}}
    \newcommand{\ram}{\gls{ram}\xspace}

\newacronym{pc}{PC}{\textit{Personal Computer}}
    \newcommand{\pc}{\gls{pc}\xspace}

\newacronym{opengl}{OpenGL}{\textit{Open Graphics Library}}
    \newcommand{\opengl}{\gls{opengl}\xspace}

\newacronym{cow}{COW}{\textit{Clusters of Workstations}}
    \newcommand{\cow}{\gls{cow}\xspace}


\newacronym{now}{NOW}{\textit{Network of Workstations}}
    \newcommand{\now}{\gls{now}\xspace}

    \newacronym{so}{SO}{\textit{Sistema Operacional}}
    \newcommand{\so}{\gls{so}\xspace}

    \newacronym{e/s}{E/S}{\textit{Entrada e Saída}}
    \newcommand{\es}{\gls{e/s}\xspace}

    \newacronym{kb}{KB}{\textit{Kilobyte}}
    \newcommand{\kb}{\gls{kb}\xspace}

    \newacronym{mb}{MB}{\textit{Megabyte}}
    \newcommand{\mb}{\gls{mb}\xspace}

    \newacronym{gb}{GB}{\textit{Gigabyte}}
    \newcommand{\gb}{\gls{gb}\xspace}

    \newacronym{posix}{POSIX}{\textit{Portable Operating System Interface}}
    \newcommand{\posix}{\gls{posix}\xspace}


\makeglossaries

\usepackage{todonotes}

\sloppy

\title{Uma Implementação do Framework PSkel com Suporte a Aplicações Estêncil
Iterativas para o Processador MPPA-256}

\author{Emmanuel Podestá Jr.\inst{1},\\Márcio Castro\inst{1}}

\address{
        Laboratório de Pesquisa em Sistemas Distribuídos (LaPeSD)\\
    Universidade Federal de Santa Catarina (UFSC) -- SC, Brasil
\email{emmanuel.podesta@grad.ufsc.br, marcio.castro@ufsc.br}}

\newcommand{\Fw}{\textit{Framework}\xspace}
\newcommand{\fw}{\textit{framework}\xspace}
\newcommand{\Fws}{\textit{Frameworks}\xspace}
\newcommand{\fws}{\textit{frameworks}\xspace}

\newcommand{\pskel}{\small \textsf{PSkel}\xspace}
\newcommand{\mppa}{\small \textsf{MPPA-256}\xspace}

\linespread{0.95}

\begin{document}

\maketitle

%\begin{abstract}
%
%\end{abstract}

\begin{resumo}
%Resumo de até 6 linhas.
Neste artigo é proposta uma adaptação do framework PSkel para o processador
\textit{manycore} \mppa. O \fw permite simplificar o desenvolvimento de
aplicações estêncil iterativas para o \mppa, escondendo do desenvolvedor
detalhes de implementação tais como a comunicação e a distribuição de
computações entre os núcleos de processamento. Os resultados mostraram um peso
significativo da comunicação no desempenho da solução.
\end{resumo}

\section{Introdução}

Diversos padrões de computação paralela são conhecidos na literatura, tais como
\textit{map}, \textit{reduce}, \textit{pipeline}, \textit{scan} e
\textit{estêncil}. Dentre eles, o padrão estêncil é um dos padrões mais
utilizados em aplicações como simulação de física de partículas, previsão
meteorológica, termodinâmica, resolução de funções diferenciais, manipulação de
imagens, entre outras~\cite{Rahman:2011:USC:2016604.2016641}. O \pskel é um \fw
de programação paralela desenvolvido para simplificar o desenvolvimento de
aplicações estêncil~\cite{pereira15}. Utilizando uma abstração de alto nível, o
programador define o \emph{kernel} da computação estêncil, enquanto o \fw se
encarrega de executar a computação paralela em \textit{multicores} e
\textit{Graphics Processing Units} (GPUs) de maneira eficiente.

Recentemente, uma adaptação preliminar do \pskel foi proposta para o processador
\mppa, um processador \textit{manycore} de baixo consumo de
energia~\cite{Castro-Podesta-ERAD:2016}. Devido às suas diversas características
peculiares e ao baixo nível de abstração requerido, desenvolver aplicações
paralelas para esse tipo de processador é uma tarefa desafiadora. Neste sentido,
a adaptação do \pskel para o \mppa permite que detalhes de baixo nível dessa
arquitetura possam ser abstraídos, além de permitir que aplicações já existentes
em \pskel possam ser portadas para essa nova plataforma sem a necessidade de
modificações de código.

O presente trabalho apresenta uma adaptação completa do \pskel para o
processador \mppa, suprindo assim as limitações da solução proposta
em~\cite{Castro-Podesta-ERAD:2016}. Essa nova versão permite: (i) execução de
aplicações estêncil iterativas; (ii) flexibilidade no particionamento dos dados;
e (iii) otimizações na computação do \textit{kernel}. Além da proposta, são
discutidos os resultados de desempenho e consumo de energia obtidos da execução
de três aplicações estêncil implementadas no \pskel. Os resultados mostram que a
forma de particionamento dos dados afeta o desempenho e que a comunicação ainda
é um fator limitante da solução proposta.

O restante desse trabalho está organizado da seguinte forma. A
Seção~\ref{sec:fundamentacao} apresenta
os principais conceitos do processador \textit{manycore} \mppa e do \fw \
\pskel.
A Seção~\ref{sec:pskelMPPA} discute a adaptação do \pskel para oferecer suporte
ao \mppa.
Os resultados são apresentados na Seção~\ref{sec:resultados} e as conclusões são
apresentadas na Seção~\ref{sec:conclusao}.

% \begin{figure}[t]
%     \begin{center}
%         \begin{tabular}{ccc}
%             \includegraphics[width=0.4\textwidth]{figs/mppa_overall.pdf} & &
% \includegraphics{figs/stencil.pdf} \\
%             (a) \mppa. & \hspace{1cm} & (b) Diagrama do padrão estêncil. \\
%         \end{tabular}
%       \vspace{-1ex}
%         \caption{Visão geral do \mppa e ilustração do padrão estêncil.}
%     \end{center}
%    \vspace{-2ex}
% \label{fig:mppa-pskel}
% \end{figure}

\section{Fundamentação Teórica}
\label{sec:fundamentacao}

%Essa seção apresenta de maneira geral o processador \mppa
%(Seção~\ref{subsec:mppa}) e o \fw \ \pskel (Seção~\ref{subsec:pskel}).


\subsection{MPPA-256}
\label{subsec:mppa}

O processador \mppa é composto por 256 núcleos de processamento de 400 MHz
denominados \pes. Como mostrado na Figura~\ref{fig:mppa-pskel}a, esses núcleos
são organizados em 16 \emph{clusters} contendo 16 \pes cada um. Cada
\textit{cluster} possui uma memória local de 2 MB (compartilhada entre todos os
\pes do \textit{cluster}) e um núcleo de sistema denominado \rman. \rmans são
responsáveis por tarefas de gerência do sistema operacional e comunicação. Além
dos \textit{clusters}, o processador apresenta 4 subsistemas de \io, sendo um
deles conectado a uma memória externa \lpddr de 2 GB. \emph{Clusters} e
subsistemas de \io se comunicam por uma \noc \textit{torus} 2D.



Estudos anteriores mostraram que desenvolver aplicações paralelas otimizadas
para o \mppa é um grande desafio~\cite{Castro-IA3-JPDC:2014} devido a alguns
fatores importantes tais como: \textbf{(i) modelo de programação híbrido}:
\textit{threads} em um mesmo \textit{cluster} se comunicam através de uma
memória compartilhada local, porém a comunicação entre \textit{clusters} é feita
explicitamente via \noc, em um modelo de memória distribuída; \textbf{(ii)
comunicação}: é necessário a utilização de uma \api específica para a
comunicação via \noc, similar ao modelo clássico POSIX de baixo nível para \ipc;
\textbf{(iii) memória}: cada \textit{cluster} possui apenas 2 MB de memória
local de baixa latência, portanto aplicações reais precisam constantemente
realizar comunicações entre o subsistema de \io (conectado à memória \lpddr); e
\textbf{(iv) coerência de \textit{cache}}: cada \pe possui uma
memória \textit{cache} privada sem coerência com as \textit{caches} dos demais
\pes, sendo necessário o uso explícito de instruções do tipo \textit{flush} para
atualizar a \textit{cache} de um \pe em determinados casos.

\subsection{PSkel}
\label{subsec:pskel}

O \pskel é um \fw de programação em alto nível para o padrão estêncil, baseado
nos conceitos de esqueletos paralelos, que oferece suporte para execução
paralela em ambientes heterogêneos incluindo CPU e GPU.  Utilizando uma única
interface de programação escrita em C++, o usuário é responsável apenas por
definir o \textit{kernel} principal da computação estêncil, enquanto o \fw se
encarrega de gerar código executável para as diferentes plataformas paralelas,
realizando de maneira transparente todo o gerenciamento de memória e
transferência de dados entre dispositivos~\cite{pereira15}.

A Figura~\ref{fig:mppa-pskel}b ilustra o funcionamento da computação estêncil em
aplicações iterativas. Em cada iteração, uma máscara de vizinhança é utilizada
na matriz de entrada para determinar o valor de cada célula da matriz de saída.
Nesse exemplo, o valor de cada célula da matriz de saída é determinado em função
dos valores das células vizinhas em todas as direções. Esse processo é realizado
para todos os pontos da matriz de entrada, produzindo uma matriz saída da
computação estêncil. Ao final de uma iteração, a matriz de saída será
considerada como sendo a matriz de entrada da próxima iteração, gerando assim
uma nova matriz de saída ao final da próxima iteração.

\section{PSkel-MPPA}
\label{sec:pskelMPPA}

A implementação do \pskel para o processador \mppa segue um modelo
mestre/escravo. Um processo mestre é executado no subsistema de \io conectado à
memória \lpddr de 2~GB, sendo responsável por alocar os dados de entrada,
distribuir as tarefas e controlar processos escravos. São criados 16 processos
escravos, um para cada \textit{cluster} de computação. Devido às limitações de
memória dos \textit{clusters}, o mestre subdivide a matriz de entrada em porções
menores denominadas \textit{tiles} e as envia para os processos escravos. O
escalonamento dos \textit{tiles} em cada iteração é feito sob demanda: cada
processo escravo recebe um \textit{tile}, realiza a computação do mesmo
utilizando o \textit{kernel} de computação estêncil definido pelo usuário e
então devolve o resultado para o mestre. A paralelização da computação dentro do
\textit{cluster} é feita com auxílio da \api OpenMP (uma \textit{thread} é
criada para cada \pe). Ao ficar ocioso, um processo escravo recebe um novo
\textit{tile} a ser computado (caso ainda existam \textit{tiles} a serem
computados). Toda a comunicação entre os processos mestre e escravos é feita
utilizando-se a \api de comunicação do \mppa.

% \begin{figure}[t]
%     \begin{center}
%         \begin{tabular}{ccc}
%             \includegraphics[width=0.314\textwidth]{figs/fur-bars-time.pdf} &
% \includegraphics[width=0.3\textwidth]{figs/gol-bars-time.pdf} &
% \includegraphics[width=0.3\textwidth]{figs/jacobi-bars-time.pdf} \\
% %           &\includegraphics[width=0.3\textwidth]{figs/bars-time-subtitle.pdf}
% %           \\
%         \end{tabular}
%       \vspace{-2ex}
%         \caption{Tempos de execução com diferentes tamanhos de \textit{tile} e
% matrizes.}\label{fig:results-bar}
%     \end{center}
%    \vspace{-2ex}
%
% \end{figure}
Para reduzir a quantidade de comunicações na NoC, é possível que um processo
escravo compute diversas iterações do estêncil antes de enviar o resultado para
o processo mestre.
Para isso, devido as dependências entre as células vizinhas do padrão estêncil,
os \textit{tiles} necessitam ser enviados juntamente com uma margem extra de
vizinhança aos escravos.
%Para isso, os \textit{tiles} necessitam ser enviados juntamente com a sua
%vizinhança aos escravos.
O tamanho da vizinhança enviado é proporcional à quantidade de iterações da
computação estêncil que poderão ser feitas no escravo sem a necessidade de
comunicação com o processo mestre. Todavia, essa técnica exige que computações
redundantes sejam feitas pelos escravos, além de aumentar a quantidade de
memória ocupada nos mesmos.

\section{Resultados}
\label{sec:resultados}

Foram utilizadas três aplicações estêncil para a realização dos experimentos. A
aplicação \textbf{Fur} tem como objetivo modelar a formação de padrões sobre a
pele de animais. A aplicação \textbf{Jacobi} implementa o método iterativo de
Jacobi para resolução de equações matriciais. Por fim, a aplicação \textbf{GoL}
é um autômato celular que implementa o Jogo da Vida de Conway. A descrição
completa dessas aplicações pode ser encontrada em~\cite{pereira15,
Castro-Podesta-ERAD:2016}.

% \begin{figure}[t]
%     \begin{center}
%         \begin{tabular}{ccc}
%             \includegraphics[width=0.35\textwidth]{figs/mppa-execTime.pdf} & &
% \includegraphics[width=0.35\textwidth]{figs/mppa-energy.pdf}
%             %(a) Tempo de execução. & \hspace{1cm} & (b) Energia.
%         \end{tabular}
%       \vspace{-2ex}
%         \caption{Desempenho e consumo de energia em três aplicações
% estêncil.}\label{fig:results-line}
%     \end{center}
%   \vspace{-2ex}
% \end{figure}

Os resultados de tempo e energia foram obtidos através de ferramentas
disponíveis no \mppa. Os experimentos foram executados sobre matrizes de entrada
de 512x512, 1024x1024, 2048x2048 e 4096x4096, alterando-se o tamanho dos
\textit{tiles} em 32x32, 64x64 e 128x128 (Figura~\ref{fig:results-bar}). Além
disso, foram feitos testes de escalabilidade, fixando-se o tamanho da matriz de
entrada em 2048x2048 e do \textit{tile} em 128x128 e alterando-se o número de
\textit{clusters} utilizados (Figura~\ref{fig:results-line}). Os valores
representam médias de 5 execuções, com um coeficiente de variação máximo
inferior à 0,4\%. Os resultados mostram um \textit{speedup} que varia entre 1,2x
e 2,7x à medida que aumenta-se o tamanho dos \textit{tiles}. Porém, a solução
proposta apresenta problemas de escalabilidade devido ao tempo de comunicação
ser muito grande em relação ao tempo total de execução da computação nos
\textit{clusters}. Esse impacto também é observado no consumo de energia obtido
quando aumenta-se o número de \textit{clusters}.

\section{Conclusão}
\label{sec:conclusao}

Neste trabalho foi proposta uma adaptação completa do \textit{framework} \pskel
para o processador \textit{manycore} \mppa. Os resultados mostraram que mesmo
obtendo resultados razoáveis ao aumentar o tamanho dos  \textit{tiles} da
computação, com um ganho de até 2,7x. A comunicação é um grande problema na
computação, tendo que ser otimizada para uma melhor utilização das
características do \mppa. Como trabalhos futuros, pretende-se otimizar a
comunicação e comparar os resultados de tempo e energia com outros processadores
(CPU e GPU).

\bibliographystyle{sbc}
\bibliography{bibliographyPaper}

\end{document}
