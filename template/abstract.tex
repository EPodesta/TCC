% ---
% RESUMOS
% ---

% resumo em português
\setlength{\absparsep}{18pt} % ajusta o espaçamento dos parágrafos do resumo
\begin{resumo}
Aplicações paralelas podem ser classificadas de acordo com o padrão de
computação e coordenação. Dentre os padrões mais conhecidos destacam-se o
\textit{map}, \textit{reduce}, \textit{pipeline}, \textit{scan} e \stencil.
Este último é muito utilizado em diversas áreas, como
simulação física partículas, previsão meteorológica, termodinâmica, resolução de
funções diferenciais, manipulação de imagens, entre outras. O \pskel é um \fw de
programação paralela desenvolvido para simplificar o desenvolvimento de
aplicações que seguem o padrão \stencil. Utilizando uma abstração de alto nível, o
programador define o \emph{kernel} da computação, enquanto o \fw se encarrega de
executar a computação paralela em \textit{multicores} e em \textit{Graphics Processing Units}
(GPUs) de maneira eficiente.
%
O objetivo deste trabalho é propor uma adaptação do \fw \pskel
para o processador \textit{manycore} emergente \mppa, batizada de \pskel-MPPA. A
motivação para tal adaptação está relacionada à dificuldade de desenvolvimento
de aplicações do padrão \textit{stencil} para o \mppa, tendo em vista as suas
características arquiteturais intrínsecas que tornam o desenvolvimento de
aplicações paralelas onerosas e suscetíveis a erros. A adaptação do \fw permite simplificar
o desenvolvimento de aplicações \stencil para o \mppa, escondendo do desenvolvedor detalhes de
implementação, tais como a comunicação e a distribuição de computações entre os
núcleos de processamento do \mppa.
%
Foram efetuados diversos experimentos com o \pskel-MPPA utilizando-se como base
três aplicações \stencil implementadas no \pskel. Os resultados mostraram que a solução
proposta apresenta boa escalabilidade no \mppa, sendo significativamente impactada
pela técnica de particionamento utilizada. Além disso, foram realizados experimentos
comparativos com a solução original do \pskel em um processador Intel Broadwell.
Os experimentos mostraram que a solução proposta no \mppa apresenta uma redução
no consumo de energia de até 1.45x com relação a solução original no Intel Broadwell,
apesar de apresentar uma perda de desempenho de até 3.3x.

 \textbf{Palavras-chave}: \textit{manycores}, \mppa, \textit{stencil}, PSkel.
\end{resumo}

%% resumo em inglês
%\begin{resumo}[Abstract]
% \begin{otherlanguage*}{english}
%Aplicações paralelas podem ser classificadas de acordo com o padrão de computação e coordenação. Dentre os padrões mais conhecidos destacam-se o \textit{map}, \textit{reduce}, \textit{pipeline}, \textit{scan} e \textit{stencil}. Este último é muito utilizado em diversas áreas, como simulação física partículas, previsão meteorológica, termodinâmica, resolução de funções diferenciais, manipulação de imagens, entre outras. O \pskel é um \fw de programação paralela desenvolvido para simplificar o desenvolvimento de aplicações que seguem esse padrão. Utilizando uma abstração de alto nível, o programador define o \emph{kernel} da computação, enquanto o \fw se encarrega de executar a computação paralela em \textit{multicores} e \textit{Graphics Processing Units} (GPUs) de maneira eficiente.
%
%O objetivo deste trabalho é propor uma adaptação do \textit{framework} \pskel para o processador \textit{manycore} emergente \mppa, batizada de \pskel-MPPA. A motivação para tal adaptação está relacionada à dificuldade de desenvolvimento de aplicações do padrão \textit{stencil} para o \mppa, tendo em vista as suas características arquiteturais intrínsecas que tornam o desenvolvimento de aplicações onerosas e suscetíveis a erros. Dentre as principais características destacam-se a sua arquitetura de memória híbrida (memória compartilha e distribuída), comunicação explícita entre processos e ausência de coerência em \textit{cache}. A adaptação do \fw permitirá simplificar o desenvolvimento de aplicações estêncil para o \mppa, escondendo do desenvolvedor detalhes de implementação, tais como a comunicação e a distribuição de computações entre os núcleos de processamento, abstraindo as características que dificultam o desenvolvimento.
%
%Serão efetuados diversos experimentos com a solução proposta para o \mppa (\pskel-MPPA) e também em outras arquiteturas suportadas pelo PSkel (\textit{multicores} e GPUs). Esses resultados permitirão a realização de um estudo comparativo dos resultados obtidos. Como métricas, serão considerados o desempenho e a eficiência energética obtidos nessas arquiteturas, além de outras possíveis métricas que possam ser interessantes para o projeto.   \vspace{\onelineskip}
%
%   \noindent
%   \textbf{Keywords}: manycores, \mppa, stencil, PSkel.
% \end{otherlanguage*}
%\end{resumo}
%
%%% resumo em francês
%%\begin{resumo}[Résumé]
%% \begin{otherlanguage*}{french}
%%    Il s'agit d'un résumé en français.
%%
%%   \textbf{Mots-clés}: latex. abntex. publication de textes.
%% \end{otherlanguage*}
%%\end{resumo}
%%
%%% resumo em espanhol
%%\begin{resumo}[Resumen]
%% \begin{otherlanguage*}{spanish}
%%   Este es el resumen en español.
%%
%%   \textbf{Palabras clave}: latex. abntex. publicación de textos.
%% \end{otherlanguage*}
%%\end{resumo}
%% ---
